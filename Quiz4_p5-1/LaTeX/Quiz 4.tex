% Type of the document
\documentclass{beamer}

% elementary packages:
\usepackage{graphicx}
\usepackage[latin1]{inputenc}
\usepackage[T1]{fontenc}
\usepackage[english]{babel}
\usepackage{listings}
\usepackage{xcolor}
\usepackage{eso-pic}
\usepackage{mathrsfs}
\usepackage{url}
\usepackage{amssymb}
\usepackage{amsmath}
\usepackage{multirow}
\usepackage{hyperref}
\usepackage{booktabs}

% additional packages
\usepackage{bbm}

% packages supplied with ise-beamer:
\usepackage{cooltooltips}
\usepackage{colordef}
\usepackage{beamerdefs}
\usepackage{lvblisting}

% Change the pictures here:
% logobig and logosmall are the internal names for the pictures: do not modify them. 
% Pictures must be supplied as JPEG, PNG or, to be preferred, PDF
\pgfdeclareimage[height=2cm]{logobig}{hulogo}
% Supply the correct logo for your class and change the file name to "logo". The logo will appear in the lower
% right corner:
\pgfdeclareimage[height=0.7cm]{logosmall}{Figures/LOB_Logo}

% Title page outline:
% use this number to modify the scaling of the headline on title page
\renewcommand{\titlescale}{1.0}
% the title page has two columns, the following two values determine the percentage each one should get
\renewcommand{\titlescale}{1.0}
\renewcommand{\leftcol}{0.6}

% Define the title.Don't forget to insert an abbreviation instead 
% of "title for footer". It will appear in the lower left corner:
\title[MSM: Quiz 4]{Selected Topics of Mathematical Statistics: Quiz 4}
% Define the authors:
\authora{Josephine Kraft} % a-c

% Define any internet addresses, if you want to display them on the title page:
\def\linka{http://lvb.wiwi.hu-berlin.de}
\def\linkb{}
\def\linkc{}
% Define the institute:
\institute{Ladislaus von Bortkiewicz Chair of Statistics \\
Humboldt--Universit\"at zu Berlin \\}

% Comment the following command, if you don't want, that the pdf file starts in full screen mode:
\hypersetup{pdfpagemode=FullScreen}

%Start of the document
\begin{document}

% Titlepage
\frame[plain]{
\titlepage
}

% Quiz 4: Task

\section{Task}

\frame{
\frametitle{Quiz 4}
\begin{center}\begin{Large}\textbf{Show several characteristic functions under \\ \vspace{2mm} different CDFs!}\end{Large}\end{center}
}

% Binomial Distribution

\section{Binomial Distribution}

\frame{
\frametitle{Binomial Distribution}
Let $X\sim \operatorname{B}(n,p)$. Then:
\begin{equation*}
\begin{align*}
\varphi_X\left(t\right)&=\operatorname{E}\left[\exp\left(\mathbf{i}tX\right)\right]=\sum_{k=0}^n \exp\left(\mathbf{i}tk\right)\cdot \operatorname{P}\left(X=k\right) \\
				    &=\sum_{k=0}^n \exp\left(\mathbf{i}tk\right)\cdot \binom{n}{k}\cdot p^k\cdot \left(1-p\right)^{n-k} \\
						&=\sum_{k=0}^n \binom{n}{k}\cdot \left\{\exp\left(\mathbf{i}t\right) \cdot p\right\}^k \cdot \left(1-p\right)^{n-k} \\
						&= \left\{\exp\left(\mathbf{i}t\right) \cdot p + \left(1-p\right)\right\}^n
\end{align*}
\end{equation*} \\ \vspace{3mm}
because $\left(a+b\right)^n=\sum_{k=0}^n \binom{n}{k}\cdot a^k\cdot b^{n-k}$ for $n\in \mathbbm{N}$.

}

% Poisson Distribution

\section{Poisson Distribution}

\frame{
\frametitle{Poisson Distribution}
Let $X\sim \operatorname{Pois}(\lambda)$. Then:
\begin{equation*}
\begin{align*}
\varphi_X\left(t\right)&=\operatorname{E}\left[\exp\left(\mathbf{i}tX\right)\right]=\sum_{k=0}^{\infty} \exp\left(\mathbf{i}tk\right) \cdot \operatorname{P}\left(X=k\right) \\
						&=\sum_{k=0}^{\infty} \exp\left(\mathbf{i}tk\right) \cdot \exp\left(-\lambda\right) \cdot \frac{\lambda^k}{k!} \\
						&= \exp\left(-\lambda\right)\cdot \sum_{k=0}^{\infty} \frac{\left\{\lambda\cdot \exp\left(\mathbf{i}t\right)\right\}^k}{k!} \\
						&= \exp\left(-\lambda\right)\cdot \exp\left(\lambda\cdot \exp\left(\mathbf{i}t\right)\right)= \exp\left(\lambda \cdot \left[\exp\left(\mathbf{i}t\right)-1\right]\right)
\end{align*}
\end{equation*}
because $\exp\left(x\right) = \sum_{k=0}^\infty \frac{x^k}{k!}$ for $x\in \mathbbm{R}$.
}

% Exponential Distribution

\section{Exponential Distribution}

\frame{
\frametitle{Exponential Distribution}
Let $X\sim \operatorname{Exp}(\lambda)$. Then:
\begin{equation*}
\begin{align*}
\varphi_X\left(t\right)&=\operatorname{E}\left[\exp\left(\mathbf{i}tX\right)\right] \\
						&=\int_0^\infty \exp\left(\mathbf{i}tx\right) \cdot f(x)\ dx \\
						&=\int_0^\infty \exp\left(\mathbf{i}tx\right) \cdot \lambda\cdot \exp\left(-\lambda x\right)\ dx \\
						&=\lambda\cdot\int_0^\infty \exp\left(\mathbf{i}tx-\lambda x\right)\  dx \\
						&=\lambda\cdot\int_0^\infty \exp\left(\left[\mathbf{i}t-\lambda\right]\cdot x\right)\  dx \\
\end{align*}
\end{equation*}

}

\frame{
\frametitle{Exponential Distribution}

\begin{equation*}
\begin{align*}
\varphi_X\left(t\right)&=\lambda\cdot\int_0^\infty \exp\left(-\left[\lambda-\mathbf{i}t\right]\cdot x\right)\  dx \\
						&=\lambda\cdot\left\{\left.-\frac{1}{\lambda-\mathbf{i}t}\cdot \exp\left(-\left[\lambda-\mathbf{i}t\right]\cdot x \right)\right\}\right|_0^\infty \\
						&=-\frac{\lambda}{\lambda-\mathbf{i}t}\cdot \left\{ \exp\left(-\left[\lambda-\mathbf{i}t\right]\cdot \infty\right) - \exp\left(-\left[\lambda-\mathbf{i}t\right]\cdot 0\right)\right\} \\
						&=-\frac{\lambda}{\lambda-\mathbf{i}t}\cdot \left\{\lim_{a\rightarrow \infty} \exp\left(-\left[\lambda-\mathbf{i}t\right]\cdot a\right) - 1\right\} \\
						&\overset{\text{(1)}}{=}-\frac{\lambda}{\lambda-\mathbf{i}t}\cdot \left(0-1\right)=\frac{\lambda}{\lambda-\mathbf{i}t}
\end{align*}
\end{equation*}

}

\frame{
\frametitle{Exponential Distribution}
To show (1), we prove that 
\begin{equation*}
\lim_{a\rightarrow \infty} \exp\left(-\left[\lambda-\mathbf{i}t\right]\cdot a\right) = 0 \eqno  (\ast)
\end{equation*}
It holds:
\begin{equation*}
\begin{align*}
\lim_{a\rightarrow\infty} \exp\left(-\left[\lambda-\mathbf{i}t\right]\cdot a\right) &= \lim_{a\rightarrow\infty} \exp\left(-\lambda a + \mathbf{i}t a\right) \\
&= \lim_{a\rightarrow\infty} \exp\left(-\lambda a\right)\cdot \exp\left(\mathbf{i}t a\right) \\
&= 0
\end{align*}
\end{equation*}
because $\lim_{a\rightarrow\infty} \exp\left(-\lambda a\right)=0$ and $|\exp\left(\mathbf{i}t a\right)|=1$. Hence, $(\ast)$ was proved and we get $\varphi_X\left(t\right)=\frac{\lambda}{\lambda-\mathbf{i}t}$.
}

% Standard Normal Distrbution

\section{Standard Normal Distribution}

\frame{
\frametitle{Standard Normal Distribution}
Let $X\sim \operatorname{N}(0,1)$. Then:
\begin{equation*}
\begin{align*}
\varphi_X\left(t\right)&=\operatorname{E}\left[\exp\left(\mathbf{i}tX\right)\right]=\int_{-\infty}^\infty \exp\left(\mathbf{i}tx\right) \cdot \frac{1}{\sqrt{2\pi}} \exp\left(-\frac{x^2}{2}\right)\ dx \\
						&= \frac{1}{\sqrt{2\pi}}\ \bigg\{\int_{-\infty}^0 \exp\left(\mathbf{i}tx\right)\cdot \exp\left(-\frac{x^2}{2}\right)\ dx \\ 
						&\qquad +\ \int_0^\infty \exp\left(\mathbf{i}tx\right)\cdot \exp\left(-\frac{x^2}{2}\right)\ dx\ \bigg\} \\
						&=\frac{1}{\sqrt{2\pi}}\ \bigg\{\int_0^\infty \exp\left(-\mathbf{i}tx\right)\cdot \exp\left(-\frac{x^2}{2}\right)\ dx \\
						&\qquad + \int_0^\infty \exp\left(\mathbf{i}tx\right)\cdot \exp\left(-\frac{x^2}{2}\right)\ dx\ \bigg\}
\end{align*}
\end{equation*}

}

\frame{
\frametitle{Standard Normal Distribution}

\begin{equation*}
\begin{align*}
\varphi_X\left(t\right)&=\frac{1}{\sqrt{2\pi}}\cdot \bigg\{\int_0^\infty \exp\left(-\frac{x^2}{2}\right)\cdot \exp\left(\mathbf{i}tx\right) \\ 
&\qquad +\ \exp\left(-\frac{x^2}{2}\right)\cdot \exp\left(-\mathbf{i}tx\right)\ dx\ \bigg\} \\
&=\frac{1}{\sqrt{2\pi}}\cdot\int_0^\infty \exp\left(-\frac{x^2}{2}\right)\cdot \left\{\exp\left(\mathbf{i}tx\right) + \exp\left(-\mathbf{i}tx\right)\right\}\ dx \\
&=\frac{1}{\sqrt{2\pi}}\cdot\int_0^\infty \exp\left(-\frac{x^2}{2}\right)\cdot 2 \cdot \operatorname{cos}\left(tx\right)\ dx \\
&=\frac{2}{\sqrt{2\pi}}\cdot\int_0^\infty \exp\left(-\frac{x^2}{2}\right)\cdot \operatorname{cos}\left(tx\right)\ dx
\overset{\text{(1)}}{=}\exp\left(-\frac{t^2}{2}\right)
\end{align*}
\end{equation*}

}

\frame{
\frametitle{Standard Normal Distribution}
To prove (1) let 
\begin{equation*}
F\left(t\right)\stackrel{\operatorname{def}}{=}\varphi_X\left(t\right)=\frac{2}{\sqrt{2\pi}}\cdot \int_0^\infty \exp\left(-\frac{x^2}{2}\right)\cdot \operatorname{cos}\left(tx\right)\ dx
\end{equation*}

We show that 
\begin{equation*}
F'\left(t\right)=-t\cdot F\left(t\right) \eqno  (\ast)
\end{equation*}
This ordinary differential equation has the solution $F\left(t\right)=c\cdot \exp\left(-\frac{t^2}{2}\right)$, where c is a constant. Since we have 
\begin{equation*}
F\left(0\right)=\int_{-\infty}^\infty \frac{1}{\sqrt{2\pi}}\cdot \exp\left(-\frac{x^2}{2}\right)\ dx = 1
\end{equation*}
we get $c=1$. Hence, $\varphi_X\left(t\right)=F\left(t\right)=\exp\left(-\frac{t^2}{2}\right)$.
}

\frame{
\frametitle{Standard Normal Distribution}

\begin{equation*}
\begin{align*}
F'\left(t\right)&=\frac{d}{dt}\ \left\{ \frac{2}{\sqrt{2\pi}}\cdot \int_0^\infty \exp\left(-\frac{x^2}{2}\right)\cdot \operatorname{cos}\left(tx\right)\  dx \right\}\\
&=\frac{2}{\sqrt{2\pi}}\cdot \int_0^\infty \exp\left(-\frac{x^2}{2}\right)\cdot \frac{d}{dt} \left\{ \operatorname{cos}\left(tx\right) \right\}\ dx \\
&=\frac{2}{\sqrt{2\pi}}\cdot \int_0^\infty -\exp\left(-\frac{x^2}{2}\right)\cdot x\cdot \operatorname{sin}\left(tx\right)\ dx
\end{align*}
\end{equation*}

Integration by parts: \\ 
$u\left(x\right)=\exp\left(-\frac{x^2}{2}\right)$ \hspace{7mm} $u'\left(x\right)=-x\cdot \exp\left(-\frac{x^2}{2}\right)$ \\ 
$v\left(x\right)=\operatorname{sin}\left(tx\right)$ \hspace{14mm} $v'\left(x\right)=t\cdot \operatorname{cos}\left(tx\right)$

}

\frame{
\frametitle{Standard Normal Distribution}

\begin{equation*}
\begin{align*}
F'\left(t\right)&=\frac{2}{\sqrt{2\pi}}\cdot \bigg[\left\{\left.\exp\left(-\frac{x^2}{2}\right)\cdot \operatorname{sin}\left(tx\right)\right\}\right|_0^\infty \\
&\qquad -\ \int_0^\infty \exp\left(-\frac{x^2}{2}\right)\cdot t \cdot \operatorname{cos}\left(tx\right)\ dx\ \bigg] \\
&=-\frac{2}{\sqrt{2\pi}}\cdot \int_0^\infty \exp\left(-\frac{x^2}{2}\right)\cdot t \cdot \operatorname{cos}\left(tx\right)\ dx \\
&=-t\cdot \int_0^\infty \frac{1}{\sqrt{2\pi}}\exp\left(-\frac{x^2}{2}\right)\cdot 2 \cdot \operatorname{cos}\left(tx\right)\ dx \\
&=-t\cdot F\left(t\right)
\end{align*}
\end{equation*}
Hence, $(\ast)$ was proved and we get $\varphi_X\left(t\right)=\exp\left(-\frac{t^2}{2}\right)$.

}

\end{document}
