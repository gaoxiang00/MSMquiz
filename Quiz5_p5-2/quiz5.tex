%----------------------------------------------------------------------------------------
%	PACKAGES AND OTHER DOCUMENT CONFIGURATIONS
%----------------------------------------------------------------------------------------

\documentclass[paper=a4, fontsize=11pt]{scrartcl} % A4 paper and 11pt font size

\usepackage[T1]{fontenc} % Use 8-bit encoding that has 256 glyphs
\usepackage{fourier} % Use the Adobe Utopia font for the document - comment this line to return to the LaTeX default
\usepackage[english]{babel} % English language/hyphenation
\usepackage{amsmath,amsfonts,amsthm} % Math packages

\usepackage{lipsum} % Used for inserting dummy 'Lorem ipsum' text into the template

\usepackage{sectsty} % Allows customizing section commands
\allsectionsfont{\centering \normalfont\scshape} % Make all sections centered, the default font and small caps

\usepackage{fancyhdr} % Custom headers and footers
\pagestyle{fancyplain} % Makes all pages in the document conform to the custom headers and footers
\fancyhead{} % No page header - if you want one, create it in the same way as the footers below
\fancyfoot[L]{} % Empty left footer
\fancyfoot[C]{} % Empty center footer
\fancyfoot[R]{\thepage} % Page numbering for right footer
\renewcommand{\headrulewidth}{0pt} % Remove header underlines
\renewcommand{\footrulewidth}{0pt} % Remove footer underlines
\setlength{\headheight}{13.6pt} % Customize the height of the header

\numberwithin{equation}{section} % Number equations within sections (i.e. 1.1, 1.2, 2.1, 2.2 instead of 1, 2, 3, 4)
\numberwithin{figure}{section} % Number figures within sections (i.e. 1.1, 1.2, 2.1, 2.2 instead of 1, 2, 3, 4)
\numberwithin{table}{section} % Number tables within sections (i.e. 1.1, 1.2, 2.1, 2.2 instead of 1, 2, 3, 4)

\setlength\parindent{0pt} % Removes all indentation from paragraphs - comment this line for an assignment with lots of text

%----------------------------------------------------------------------------------------
%	TITLE SECTION
%----------------------------------------------------------------------------------------

\newcommand{\horrule}[1]{\rule{\linewidth}{#1}} % Create horizontal rule command with 1 argument of height

\title{	
\normalfont \normalsize 
\textsc{Humboldt University Berlin} \\ [25pt] % Your university, school and/or department name(s)
\horrule{0.5pt} \\[0.4cm] % Thin top horizontal rule
\huge Quiz 5 \\ % The assignment title
\horrule{2pt} \\[0.5cm] % Thick bottom horizontal rule
}

\author{Malte Esders} % Your name

\date{\normalsize\today} % Today's date or a custom date

\begin{document}

\maketitle % Print the title

\section{Problem Description}
Quiz 5: Prove (18) under standard normal distribution, where (18) is:

If $\varphi_X$ is absolutely integrable,
\begin{align} 
\begin{split}
	(18)\  f_X(x) = \frac{1}{(2\pi)^p}\ \int_{-\infty}^{\infty}{e^{-it^\intercal x} \varphi_X(t)\ dt}
\end{split}					
\end{align}

\newpage

\section{Solution}


To prove the statement, first we will need one Lemma:

\begin{equation}
	\int_{-\infty}^{\infty}{e^{-\frac{1}{2}(x-it)^2}\ dx} = \sqrt{2\pi}
\end{equation}

\subsection{Proof of Lemma 1}
We start with
\begin{equation}
	\int_{-\infty}^{\infty}{e^{-\frac{1}{2}(x-it)^2}\ dx}
\end{equation}
\\

First, substitute $S = x-it$, we have

\begin{equation}
\begin{aligned}
	\int_{-\infty-it}^{\infty-it}{e^{-\frac{1}{2}S^2}\ dS}
\end{aligned}
\end{equation}

Now, remember that, on a closed contour if the function within the contour is analytic,
\begin{equation}
	\oint{f(z)\ dz} = 0
\end{equation}


We're first taking the integral between $\alpha$ and $-\alpha$ and later take the limits at $\infty$.
Consider the integral on a contour like this: $\mathcal{C} = \alpha \rightarrow -\alpha \rightarrow -\alpha-it \rightarrow \alpha-it \rightarrow \alpha.$ \\
Now since the normal distribution is analytic everywhere, we must have
\begin{equation}
	\oint_\mathcal{C}{f_X(z)\ dz} = 0
\end{equation}
\\
Writing out all four parts of the contour integral gives
\begin{equation}
	\oint_\mathcal{C}{f(S)\ dS} = \int_{\alpha}^{-\alpha}{e^{-\frac{S^2}{2}}\ dS} + \int_{-\alpha}^{-\alpha-it}{e^{-\frac{S^2}{2}}\ dS} + \int_{-\alpha-it}^{\alpha-it}{e^{-\frac{S^2}{2}}\ dS} + \int_{\alpha-it}^{\alpha}{e^{-\frac{S^2}{2}}\ dS} = 0
\end{equation}
\\
As we take the limits for $\alpha \rightarrow \infty$, the first term becomes $-\sqrt{2\pi}$ (because we're integrating from right to left), and terms 2 and 4 become zero. The third term is the term we're interested in. As we solve for that term, we get
\begin{equation}
	\int_{-\infty-it}^{\infty-it}{e^{-\frac{S^2}{2}}\ dS} = \sqrt{2\pi}
\end{equation}
which completes the proof.

\subsection{Proof of (18)}
In order to proof (18), we first compute the characteristic function $\varphi_X(t)$ of the standard normal distribution. From (17) we have
\begin{equation}
	\varphi_X(t) = \int_{-\infty}^{\infty}{e^{itx} \frac{1}{\sqrt{2\pi}} e^{-\frac{x^2}{2}}\ dx}
\end{equation}
\\
Looking at the exponent of $e$, we complete the square in t
\begin{equation}
	\begin{aligned}
	\varphi_X(t) &= \frac{1}{\sqrt{2\pi}} \int_{-\infty}^{\infty}{e^{-\frac{1}{2} (x^2 +2itx -t^2)}  e^{-\frac{1}{2}t^2}\ dx} \\
	&= \frac{1}{\sqrt{2\pi}} e^{-\frac{1}{2}t^2}\ \int_{-\infty}^{\infty}{e^{-\frac{1}{2} (x^2 +2itx -t^2)}} \\
	&= \frac{1}{\sqrt{2\pi}} e^{-\frac{1}{2}t^2}\ \int_{-\infty}^{\infty}{e^{-\frac{1}{2} (x - it)^2}}
	\end{aligned}
\end{equation}
\\
Now by Lemma 1, the Integral in the last line of (2.9) is $\sqrt{2\pi}$, and $\varphi_X(t)$ is
\begin{equation}
	\varphi_X(t) = e^{-\frac{t^2}{2}}
\end{equation}
\\
To complete the proof we substitute $\varphi_X(t)$ into (18):
\begin{equation}
	f_X(x) = \frac{1}{2\pi}\ \int_{-\infty}^{\infty}{e^{-itx} e^{-\frac{t^2}{2}}\ dt}
\end{equation}
\\
We proceed by completing the squre similarly to (2.9) and get
\begin{equation}
	f_X(x) = \frac{1}{2\pi} e^{-\frac{x^2}{2}}\ \int_{-\infty}^{\infty}{e^{-\frac{1}{2} (t - ix)^2}\ dt}
\end{equation}
\\
Again by Lemma 1, the integral is $\sqrt{2\pi}$ and we are left with
\begin{equation}
	f_X(x) = \frac{1}{\sqrt{2\pi}} e^{-\frac{x^2}{2}}\ 
\end{equation}
which completes the proof.


\end{document}

%%%%%%%%%%%%%%%%%%%%%%%%%%%%%%%%%%%%%%%%%
% Short Sectioned Assignment
% LaTeX Template
% Version 1.0 (5/5/12)
%
% This template has been downloaded from:
% http://www.LaTeXTemplates.com
%
% Original author:
% Frits Wenneker (http://www.howtotex.com)
%
% License:
% CC BY-NC-SA 3.0 (http://creativecommons.org/licenses/by-nc-sa/3.0/)
%
%%%%%%%%%%%%%%%%%%%%%%%%%%%%%%%%%%%%%%%%%

