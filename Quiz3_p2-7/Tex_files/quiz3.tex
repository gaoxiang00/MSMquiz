% Type of the document
\documentclass{beamer}

% elementary packages:
\usepackage{graphicx}
\usepackage[latin1]{inputenc}
\usepackage[T1]{fontenc}
\usepackage[english]{babel}
\usepackage{listings}
\usepackage{xcolor}
\usepackage{eso-pic}
\usepackage{mathrsfs}
\usepackage{url}
\usepackage{amssymb}
\usepackage{amsmath}
\usepackage{multirow}
\usepackage{hyperref}
\usepackage{booktabs}
\usepackage{dsfont}
% additional packages
\usepackage{bbm}

% packages supplied with ise-beamer:
\usepackage{cooltooltips}
\usepackage{colordef}
\usepackage{beamerdefs}
\usepackage{lvblisting}

% Change the pictures here:
% logobig and logosmall are the internal names for the pictures: do not modify them. 
% Pictures must be supplied as JPEG, PNG or, to be preferred, PDF
\pgfdeclareimage[height=2cm]{logobig}{hulogo}
% Supply the correct logo for your class and change the file name to "logo". The logo will appear in the lower
% right corner:
\pgfdeclareimage[height=0.7cm]{logosmall}{Figures/LOB_Logo}

% Title page outline:
% use this number to modify the scaling of the headline on title page
\renewcommand{\titlescale}{1.0}
% the title page has two columns, the following two values determine the percentage each one should get
\renewcommand{\titlescale}{1.0}
\renewcommand{\leftcol}{0.6}

% Define the title.Don't forget to insert an abbreviation instead 
% of "title for footer". It will appear in the lower left corner:
\title[Quiz 3]{Selected topics in Mathematical Statistics, Quiz 3}
% Define the authors:
\authora{Malte Esders} % a-c
\authorb{}
\authorc{}

% Define any internet addresses, if you want to display them on the title page:
\def\linka{http://lvb.wiwi.hu-berlin.de}
\def\linkb{}
\def\linkc{}
% Define the institute:
\institute{Humboldt--Universit�t zu Berlin \\}

% Comment the following command, if you don't want, that the pdf file starts in full screen mode:
%\hypersetup{pdfpagemode=FullScreen}

%Start of the document
\begin{document}

% create the title slide, layout controlled in beamerdefs.sty and the foregoing specifications
\frame[plain]{
\titlepage
}

%%%%%%%%%%%%%%%%%%%%%%%%%%%%%%%%%%%%%%%%%%%%%%%%%%%%%%%%%%%%%%%%%%%%%%%%%%%%%%%%%%%%%%%%%%%%%%%%%%%%%%%%%%%%%%%%%%%%%%%%
\section{Problem Description}

\frame{
\frametitle{Problem Description}
Quiz 3: Show that the Kullback-Leibler-Divergence $K(\theta,\theta')$ satisfies for any $\theta$, $\theta'$:
\begin{align} 
\begin{split}
	&1.\ K(\theta,\theta')|_{\theta=\theta'} = 0 \\
	&2.\ \frac{d}{d\theta'}\ K(\theta,\theta')|_{\theta=\theta'} = 0 \\
	&3.\ \frac{d^2}{d\theta'^2}\ K(\theta,\theta')|_{\theta=\theta'} = \int_{-\infty}^{\infty}{\frac{p'(x, \theta)^2}{p(x, \theta)}\ dx} \\
\end{split}					
\end{align}

}


%%%%%%%%%%%%%%%%%%%%%%%%%%%%%%%%%%%%%%%%%%%%%%%%%%%%%%%%%%%%%%%%%%%%%%%%%%%%%%%%%%%%%%%%%%%%%%%%%%%%%%%%%%%%%%%%%%%%%%%%
\section{Statement 1}

\frame{
\frametitle{Proof of statement 1}
Statement 1 was:

\begin{align} 
	K(\theta,\theta')|_{\theta=\theta'} = 0 \\
\end{align}
Notice that the quantity in the logarithm is always one, therefore the logarithm evaluates to zero:
\begin{align} 
\begin{split}
	K(\theta,\theta')|_{\theta=\theta'} = \int_{-\infty}^{\infty}{\ln{1}\ p(x, \theta)dx} = \int_{-\infty}^{\infty}{0*p(x, \theta)dx} = 0
\end{split}					
\end{align}

}

\section{Statement 2}

\frame{
\frametitle{Proof of statement 2}
Statement 2 was:
\begin{align} 
	\frac{d}{d\theta'}\ K(\theta,\theta')|_{\theta=\theta'} = 0 \\
\end{align}
For this proof we'll assume that $p(x,\theta) = p(x,\theta')$ and its derivative with respect to $\theta$ is continuous over $\mathds{R}$, in which case we can evaluate the derivative within the integral according to Leibniz' rule.
}

\frame{
Use Leibniz' rule:
\begin{equation}
\begin{aligned} 
	\frac{d}{d\theta'}\ K(\theta,\theta')|_{\theta,\theta'} &= \frac{d}{d\theta'}\ \int_{-\infty}^{\infty}{\ln{\frac{p(x,\theta)}{p(x,\theta')}}\ p(x, \theta)dx} \\
	&= \int_{-\infty}^{\infty}{p(x, \theta)\ \frac{d}{d\theta'} \Bigg(\ln{\frac{p(x,\theta)}{p(x,\theta')}}\Bigg) dx} \\
\end{aligned}
\end{equation}

$p(x,\theta)$ is independent of $\theta'$, so by rewriting the logarithm we get
\begin{equation}
\begin{aligned} 
	&= \int_{-\infty}^{\infty}p(x, \theta){\ \frac{d}{d\theta'} \Big\{ -\ln{(p(x,\theta')}) \Big\} \ dx} \\
\end{aligned}
\end{equation}

}

\frame{
Evaluating the derivate further:
\begin{equation}
\begin{aligned} 
	&= \int_{-\infty}^{\infty}{-p(x, \theta)\ \frac{1}{p(x,\theta')}\ \frac{d}{d\theta'} p(x,\theta')\ dx} \\
\end{aligned}
\end{equation}
Since $\theta = \theta'$, the first two factors cancel
\begin{equation}
\begin{aligned} 
	&= \int_{-\infty}^{\infty}{-\frac{d}{d\theta'} p(x,\theta')\ dx} \\
	&= \frac{d}{d\theta'} \int_{-\infty}^{\infty}{-p(x,\theta')\ dx} \\
\end{aligned}
\end{equation}
}


\frame{
and the integral is over a probability distribution, so
\begin{equation}
\begin{aligned} 
	&= \frac{d}{d\theta'} \int_{-\infty}^{\infty}{-p(x,\theta')\ dx} \\
	&= \frac{d}{d\theta'} (-1) \\
	&= 0
\end{aligned}
\end{equation}
}


\section{Statement 3}
\frame{
\frametitle{Proof of statement 3}
Similar to part 2.2, we assume now that also the second derivative of $p(x,\theta)$ with respect to $\theta$ is continuous, and use Leibniz rule:
\begin{equation}
\begin{aligned} 
	\frac{d^2}{d\theta'^2}\ K(\theta,\theta')|_{\theta,\theta'} &= \frac{d^2}{d\theta'^2} \int_{-\infty}^{\infty}{\ln{\frac{p(x,\theta)}{p(x,\theta'}}\ p(x, \theta)dx} \\
	&= \int_{-\infty}^{\infty}{p(x, \theta)\ \frac{d^2}{d\theta'^2}\ \Bigg\{ \ln{\frac{p(x,\theta)}{p(x,\theta')}}\Bigg\} \ dx} \\
\end{aligned}
\end{equation}
}

\frame{
again making use of the fact that $p(x,\theta)$ is independent of $\theta'$, and evaluating the derivative:
\begin{equation}
\begin{aligned} 
	&= \int_{-\infty}^{\infty}{p(x, \theta)\ \frac{d^2}{d\theta'^2}\ \Bigg\{ \ln{\frac{p(x,\theta)}{p(x,\theta')}}\Bigg\} \ dx} \\
	&= \int_{-\infty}^{\infty}{-p(x, \theta)\ \frac{d^2}{d\theta'^2}\ \ln{p(x,\theta')}\ dx} \\
	&= \int_{-\infty}^{\infty}{-p(x, \theta)\ \frac{d}{d\theta'}\ \Bigg[ \frac{1}{p(x,\theta')}\ \frac{d}{d\theta'} p(x,\theta')\Bigg]\ dx} \\
\end{aligned}
\end{equation}
}

\frame{
and evaluating the derivative again
\begin{equation}
\begin{aligned} 
	&= \int_{-\infty}^{\infty}{-p(x, \theta)\ \frac{d}{d\theta'}\ \Bigg[ \frac{1}{p(x,\theta')}\ \frac{d}{d\theta'} p(x,\theta')\Bigg]\ dx} \\
	&= \int_{-\infty}^{\infty}-p(x, \theta)\ \Bigg[ -\frac{1}{p(x,\theta')^2}\ \Bigg\{ \frac{d}{d\theta'} p(x,\theta') \Bigg\}^2 \\
	&+ \ \ \ \ \ \ \frac{1}{p(x,\theta')}\  \frac{d^2}{d\theta'^2}\ p(x,\theta') \Bigg]\ dx \\
\end{aligned}
\end{equation}
multiplying by $-p(x,\theta)$ in front of the brackets:
\begin{equation}
\begin{aligned} 
	&= \int_{-\infty}^{\infty}{\frac{\Bigg\{ \frac{d}{d\theta'} p(x,\theta') \Bigg\}^2}{p(x,\theta')}  -  \frac{d^2}{d\theta'^2}\ p(x,\theta')\ dx} \\
\end{aligned}
\end{equation}
}

\frame{

\begin{equation}
\begin{aligned} 
	&= \int_{-\infty}^{\infty}{\frac{\Bigg\{ \frac{d}{d\theta'} p(x,\theta') \Bigg\}^2}{p(x,\theta')}\ dx}  -  \frac{d^2}{d\theta'^2}\ \int_{-\infty}^{\infty}{p(x,\theta')\ dx} \\
	&= \int_{-\infty}^{\infty}{\frac{\Bigg\{ \frac{d}{d\theta'} p(x,\theta') \Bigg\}^2}{p(x,\theta')}\ dx}  - 0 \\
	&= \int_{-\infty}^{\infty}{\frac{\Bigg\{ \frac{d}{d\theta'} p(x,\theta') \Bigg\}^2}{p(x,\theta')}\ dx} \\
\end{aligned}
\end{equation}
}

\frame{
In a different notation, this is:
\begin{equation}
	= \int_{-\infty}^{\infty}\frac{p'(x, \theta)^2}{p(x, \theta)}\ dx 
\end{equation}
which proves statement three.
}

% Define the end of the document:
\end{document}
