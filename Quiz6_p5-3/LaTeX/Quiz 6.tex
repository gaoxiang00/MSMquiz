% Type of the document
\documentclass{beamer}

% elementary packages:
\usepackage{graphicx}
\usepackage[latin1]{inputenc}
\usepackage[T1]{fontenc}
\usepackage[english]{babel}
\usepackage{listings}
\usepackage{xcolor}
\usepackage{eso-pic}
\usepackage{mathrsfs}
\usepackage{url}
\usepackage{amssymb}
\usepackage{amsmath}
\usepackage{multirow}
\usepackage{hyperref}
\usepackage{booktabs}

% additional packages
\usepackage{bbm}

% packages supplied with ise-beamer:
\usepackage{cooltooltips}
\usepackage{colordef}
\usepackage{beamerdefs}
\usepackage{lvblisting}

% Change the pictures here:
% logobig and logosmall are the internal names for the pictures: do not modify them. 
% Pictures must be supplied as JPEG, PNG or, to be preferred, PDF
\pgfdeclareimage[height=2cm]{logobig}{hulogo}
% Supply the correct logo for your class and change the file name to "logo". The logo will appear in the lower
% right corner:
\pgfdeclareimage[height=0.9cm]{logosmall}{Figures/msm}

% Title page outline:
% use this number to modify the scaling of the headline on title page
\renewcommand{\titlescale}{1.0}
% the title page has two columns, the following two values determine the percentage each one should get
\renewcommand{\titlescale}{1.0}
\renewcommand{\leftcol}{0.6}

% Define the title.Don't forget to insert an abbreviation instead 
% of "title for footer". It will appear in the lower left corner:
\title[MSM: Quiz 6]{Selected Topics of Mathematical Statistics: Quiz 6}
% Define the authors:
\authora{Josephine Kraft} % a-c

% Define any internet addresses, if you want to display them on the title page:
\def\linka{http://lvb.wiwi.hu-berlin.de}
\def\linkb{}
\def\linkc{}
% Define the institute:
\institute{Ladislaus von Bortkiewicz Chair of Statistics \\
Humboldt--Universit\"at zu Berlin \\}

% Comment the following command, if you don't want, that the pdf file starts in full screen mode:
\hypersetup{pdfpagemode=FullScreen}

%Start of the document
\begin{document}

% Titlepage
\frame[plain]{
\titlepage
}

% Quiz 6: Task

\section{Task}

\frame{
\frametitle{Quiz 6}

\begin{Large}\noindent \textbf{Prove the following property of the characteristic function for the Standard Normal Distribution for $k=1,2,3,4$:}
\begin{equation*}
\varphi^{\left(k\right)}_{X}\left(t\right) |_{t=0}=\mathbf{i}^{k}\cdot \operatorname{E}\left[X^k\right]
\end{equation*}
\end{Large}

}

\section{Proof}

\frame{
\frametitle{Characteristic Function of the Standard Normal Distribution}

In Quiz 4 it was shown that the characteristic function of a standard normal random variable is of the following form:

\begin{equation*}
\varphi_X\left(t\right)=\exp\left(-\frac{t^2}{2}\right),\ \hspace{3mm}t\in\mathbbm{R}
\end{equation*}

}

% k=1

\section{Proof: k=1}

\frame{
\frametitle{Proof for k=1}

\begin{equation*}
\begin{align*}
\varphi^{\left(1\right)}_{X}\left(t\right) |_{t=0}	&=\frac{d}{dt} \left.\left\{\exp\left(-\frac{t^2}{2}\right)\right\}\right|_{t=0} \\
														&=\left.\left\{-t \exp\left(-\frac{t^2}{2}\right)\right\}\right|_{t=0} \\
														&=0 \\
														&=\mathbf{i}\cdot \operatorname{E}\left[X\right]
\end{align*}
\end{equation*} \\ \vspace{3mm}
because $\operatorname{E}\left[X\right]=0$ for the Standard Normal Distribution.

}

% k=2

\section{Proof: k=2}

\frame{
\frametitle{Proof for k=2}

\begin{equation*}
\begin{align*}
\varphi^{\left(2\right)}_{X}\left(t\right) |_{t=0}	&=\frac{d^2}{dt^2} \left.\left\{\exp\left(-\frac{t^2}{2}\right)\right\}\right|_{t=0} \\
														&=\frac{d}{dt} \left.\left\{-t \exp\left(-\frac{t^2}{2}\right)\right\}\right|_{t=0} \\
														&=\left.\left\{-\exp\left(-\frac{t^2}{2}\right) + t^2 \exp\left(-\frac{t^2}{2}\right)\right\}\right|_{t=0} \\
														&=-1 \\
														&=\mathbf{i}^2\cdot \operatorname{Var}\left(X\right) 
														=\mathbf{i}^2\cdot \operatorname{E}\left[X^2\right]
\end{align*}
\end{equation*} \\ \vspace{3mm}
because $\operatorname{Var}\left(X\right)=1$ and $\operatorname{E}\left[X\right]=0$ for the Standard Normal Distribution.

}

% k=3

\section{Proof: k=3}

\frame{
\frametitle{Proof for k=3}

\begin{equation*}
\begin{align*}
\varphi^{\left(3\right)}_{X}\left(t\right) |_{t=0}	&=\frac{d^3}{dt^3} \left.\left\{\exp\left(-\frac{t^2}{2}\right)\right\}\right|_{t=0} \\
														&=\frac{d}{dt} \left.\left\{-\exp\left(-\frac{t^2}{2}\right) + t^2 \exp\left(-\frac{t^2}{2}\right)\right\}\right|_{t=0} \\
														&=\left.\left\{t\exp\left(-\frac{t^2}{2}\right) + 2t\exp\left(-\frac{t^2}{2}\right) - t^3\exp\left(-\frac{t^2}{2}\right)\right\}\right|_{t=0} \\
														&=0 \\
														&=\mathbf{i}^3\cdot \operatorname{E}\left[X^3\right]
\end{align*}
\end{equation*}\\
\\
because $\operatorname{E}\left[X^3\right]=0$ for the Standard Normal Distribution.

}

% k=4

\section{Proof: k=4}

\frame{
\frametitle{Proof for k=4}

\begin{equation*}
\begin{align*}
\varphi^{\left(4\right)}_{X}\left(t\right) |_{t=0}	&=\frac{d^4}{dt^4} \left.\left\{\exp\left(-\frac{t^2}{2}\right)\right\}\right|_{t=0} \\
														&=\frac{d}{dt} \bigg\{t \exp\left(-\frac{t^2}{2}\right) + 2t\exp\left(-\frac{t^2}{2}\right) \\
														&\qquad \left.- t^3 \exp\left(-\frac{t^2}{2}\right)\bigg\}\right|_{t=0} \\
														&=\frac{d}{dt} \left.\left\{t \exp\left(-\frac{t^2}{2}\right)\right\}\right|_{t=0} + \frac{d}{dt} \left.\left\{2t \exp\left(-\frac{t^2}{2}\right)\right\}\right|_{t=0} \\
														&\qquad - \frac{d}{dt} \left.\left\{t^3 \exp\left(-\frac{t^2}{2}\right)\right\}\right|_{t=0}
\end{align*}
\end{equation*}

}

\frame{
\frametitle{Proof for k=4}

It holds:
\begin{equation}
\begin{align}
\frac{d}{dt} \left.\left\{t \exp\left(-\frac{t^2}{2}\right)\right\}\right|_{t=0}&=\left.\left\{\exp\left(-\frac{t^2}{2}\right) - t^2 \exp\left(-\frac{t^2}{2}\right)\right\}\right|_{t=0} \\
&= 1
\end{align}
\end{equation}

\begin{equation}
\begin{align}
\frac{d}{dt} \left.\left\{2t\exp\left(-\frac{t^2}{2}\right)\right\}\right|_{t=0}&=2\left.\left\{\exp\left(-\frac{t^2}{2}\right) - t^2 \exp\left(-\frac{t^2}{2}\right)\right\}\right|_{t=0} \\
&= 2
\end{align}
\end{equation}
}

\frame{
\frametitle{Proof for k=4}

\begin{equation}
\begin{align}
\frac{d}{dt} \left.\left\{t^3\exp\left(-\frac{t^2}{2}\right)\right\}\right|_{t=0}&=\left.\left\{3 t^2\exp\left(-\frac{t^2}{2}\right) - t^4\exp\left(-\frac{t^2}{2}\right)\right\}\right|_{t=0} \\
&= 0
\end{align}
\end{equation}

With (1), (2), (3) it follows:

\begin{equation*}
\begin{align*}
\varphi^{\left(4\right)}_{X}\left(t\right) |_{t=0}	&=1+2-0 \\
															&=3\\
															&=\mathbf{i}^4\cdot \operatorname{E}\left[X^4\right]
\end{align*}
\end{equation*}
because $\mathbf{i}^4=\mathbf{i}^2\cdot \mathbf{i}^2=(-1)\cdot(-1)=1$ and $\operatorname{E}\left[X^4\right]=3$ for the Standard Normal Distribution.

}

\end{document}