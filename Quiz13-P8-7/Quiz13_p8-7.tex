% Type of the document
\documentclass{beamer}

% elementary packages:
\usepackage{graphicx}
\usepackage[latin1]{inputenc}
\usepackage[T1]{fontenc}
\usepackage[english]{babel}
\usepackage{listings}
\usepackage{xcolor}
\usepackage{eso-pic}
\usepackage{mathrsfs}
\usepackage{url}
\usepackage{amssymb}
\usepackage{amsmath}
\usepackage{multirow}
\usepackage{hyperref}
\usepackage{booktabs}

% additional packages
\usepackage{bbm}

% packages supplied with ise-beamer:
\usepackage{cooltooltips}
\usepackage{colordef}
\usepackage{beamerdefs}
\usepackage{lvblisting}
% Pictures must be supplied as JPEG, PNG or, to be preferred, PDF
\pgfdeclareimage[height=4.5cm]{logobig}{LogoIRTG}
% Supply the correct logo for your class and change the file name to "logo". The logo will appear in the lower
% right corner:
\pgfdeclareimage[height=0.7cm]{logosmall}{MSM_AN.png}

% Title page outline:
% use this number to modify the scaling of the headline on title page
\renewcommand{\titlescale}{1.0}
% the title page has two columns, the following two values determine the percentage each one should get
\renewcommand{\titlescale}{1.0}
\renewcommand{\leftcol}{0.6}

% Define the title.Don't forget to insert an abbreviation instead 
% of "title for footer". It will appear in the lower left corner:
\title[Solutions to Quizzes]{Solutions to Quizzes}
% Define the authors:
\authora{Ya Qian} % a-c
\begin{document}
% Define any internet addresses, if you want to display them on the title page:
\def\linka{lvb.wiwi.hu-berlin.de}
\def\linkb{\\case.hu-berlin.de}
\def\linkc{\\irtg1792.hu-berlin.de}
% Define the institute:
\institute{Ladislaus von Bortkiewicz Chair of Statistics\\
C.A.S.E. -- Center for Applied Statistics\\
and Economics\\
Humboldt--Universit\"at zu Berlin\\}

% Comment the following command, if you don't want, that the pdf file starts in full screen mode:
\hypersetup{pdfpagemode=FullScreen}

%Start of the document
\begin{document}

% create the title slide, layout controlled in beamerdefs.sty and the foregoing specifications
\frame[plain]{
\titlepage
}

%%%%%%%%%%%%%%%%%%%%%%%%%%%%%%%%%%%%%%%%%%%%%%%%%%%%%%%%%%%%%%%%%%%%%%%%%%%%%%%%%%%%%%%%%%%%%%%%%%%%%%%%%%%%%%%%%%%%%%%%

\section{Solution to Quizzes}
\frame[plain]{
\vspace{-2cm}
\LARGE
\color{isered}
\textbf{
\begin{center}
Quiz 13:\\
Derive the above m.f. for $X \sim \operatorname{N}(\mu,{\sigma}^2)$ and $X \sim \operatorname{B}(1,p)$}
\end{center}
}

\frame{
\frametitle{Solution to Quiz 13}
\begin{itemize}
\item For continuous $r.v. X \in \mathbbm{R}$, moment generating function is $M_X(t)=\operatorname{E}[e^{tX}]=\int_{-\infty}^{\infty}e^{tx}f(x)dx$:\\
Therefore for $X \sim \operatorname{N}(\mu,{\sigma}^2)$, we have \\
\begin{eqnarray*}
M_X(t)=\operatorname{E}[e^{tX}]
&=&\int_{-\infty}^{\infty}e^{tx}\cdot \frac{1}{\sqrt{2\pi} \sigma}e^{-\frac{(x-\mu)^2}{2 {\sigma}^2}}dx\\
&=&\int_{-\infty}^{\infty}\frac{1}{\sqrt{2\pi} \sigma}e^{-\frac{[x-(\mu+ t{\sigma}^2)]^2}{2\sigma^2}}\cdot e^{(t\mu +t^2{\sigma}^2/2)}dx\\
&=&1\cdot \exp(t\mu+ t^2\sigma^2/2)=\exp(t\mu+ t^2{\sigma}^2/2)
\end{eqnarray*}
\end{itemize}
}

\frame{
\frametitle{Solution to Quiz 13}
\begin{itemize}
\item For discrete $r.v. X \in \mathbbm{Z}$, moment generating function is $M_X(t)=\sum_i {e^{tx_i}\cdot p_i}$\\
Therefore for $X \sim \operatorname{B}(1,p)$, we have \\
\begin{eqnarray*}
M_X(t)=e^{t\cdot 0}\cdot (1-p)+e^{t\cdot 1}\cdot p =1-p+pe^t
\end{eqnarray*}


\end{itemize}
}
\end{document}